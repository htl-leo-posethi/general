\documentclass{beamer}
%%% BEGIN PREAMBLE

\usepackage{amsmath}
\usepackage{amsfonts}
% \usepackage{beamerthemesplit} // Activate for custom appearance
\usepackage{graphicx}
\usepackage[utf8]{inputenc}

% Definitions
\theoremstyle{remark}
\newtheorem{remark}[theorem]{Remark}

% Title stuff
\title{Unit 05 – Proofs}
\author{Peter Bauer}
\date{} % delete this line to display the current date

%%% BEGIN DOCUMENT
\begin{document}

\frame{\titlepage}

\begin{frame}
	\frametitle{Outline}
	\tableofcontents
\end{frame}

\section{Axioms}

\begin{frame}
 \frametitle{Axioms}
 
\begin{definition}
 {\em Axioms} are basic propositions or propositional formulas which are true (hold) from the beginning. They are normally accepted without any proof.
\end{definition}

\begin{example}
\begin{itemize}
	\item An equivalence relation $R$ is reflexive, symmetric, and transitive.
	\item The composition $\land$ is associative, commutative, and distributive.
\end{itemize}
\end{example}
\end{frame}

\begin{frame}
\frametitle{Proof Rules}
Proof rules determine how you can make new true formulas from already existing true formulas.
\end{frame}

\section{Direct Proof}
\begin{frame}
\frametitle{Direct Proof}
\begin{itemize}
 	\item Step-by-step conclusion
	\item $(A \Rightarrow C) \land (C \Rightarrow B) \vdash (A \Rightarrow B)$
\end{itemize}
\pause
\begin{example}
 
\begin{enumerate}
 \item If you own a longhair cat it will lose hairs
 \pause
 \item If somebody/something loses hairs, your couch will be (sooner or later) contaminated with hairs
 \pause
 \item If you own a longhair cat your couch will be (sooner or later) contaminated with hairs
\end{enumerate}
\end{example}

\pause 

\begin{remark}
	A simplified but famous version of the direct proof is the so-called {\em Modus Ponens}:
	$A \land (A \Rightarrow B) \vdash B$
\end{remark}
\end{frame}

\section{Indirect Proof}
\begin{frame}
\frametitle{Indirect Proof (Modus Tollens)}
	
\begin{itemize}
	\item $A \Rightarrow B$ has to be proven
	\pause
	\item Sometimes this turns out to be hard or even impossible to be proven
	\pause
	\item Therefore, we assume that $\lnot B$
	\pause
	\item Then we try to derive $\lnot A$
	\pause
	\item $A \Rightarrow B \equiv \lnot B \Rightarrow \lnot A$ 
\end{itemize}

\pause
\begin{example}
\begin{itemize}
	 \item If it is the last school day of the year we get our final certificates.
	 \pause
	 \item This is logically equivalent to
	 \pause
	 \item If we do not get our final certificates it can't be the last school day of the year.
\end{itemize}
\end{example}
\pause
Prove the rule via truth table.
\end{frame}

\begin{frame}
\frametitle{Modus Tollens -- A More Formal Example}
\begin{example}
\begin{itemize}
	 \item $a \cdot b$ is even $\Rightarrow a$ is even or $b$ is even.
	 \pause
	 \item Assume that $a$ is odd and $b$ is odd
	 \pause
	 \item Show under this assumption that $a \cdot b$ must be odd
	 %a . b = 2k => a = 2l or b = 2m. Assume a = 2l + 1 and b = 2m + 1. Show that there exists a k such that a . b == 2k + 1
\end{itemize}
\end{example}
\end{frame}

\section{Proof by Contradiction}
\begin{frame}
\frametitle{Proof by Contradiction}
\begin{itemize}
	\item $A \Rightarrow B$ has to be proven	
	\pause
	\item Assume $A$
	\pause
	\item Assume $\lnot B$
	\pause
	\item Then we try to proof a contradiction
	\pause
	\item $A \Rightarrow B \equiv A \land \lnot B \Rightarrow \text{{\bf F}}$
	\pause
	\item Prove this rule via a truth table
\end{itemize}
\pause
\begin{example}
\begin{itemize}
	\item Prove that $a^2$ is even $\Rightarrow a$ is even
	\item Assume that $a^2$ is even
	\item Assume for the sake of contradiction that $a$ is odd
	\item Then you can conclude that $a^2$ is odd, which is in contradiction to our first assumption
	% If a should be odd it must be 2k + 1 for some arbitrary k. Then we could say that a^2 = (2k + 1)^2 = 4k^2 + 4k + 1 = 4k(4k + 1) + 1. Obviously 4k(k + 1) must be even (4 * x is even for any x), so 4k(4k + 1) + 1 must be odd which means that a^2 must be odd, which is in contradiction to our assumption that a^2 is even. 
\end{itemize}
\end{example}
\end{frame}

\begin{frame}
	\frametitle{Proof by Contradiction}
	\framesubtitle{A Specific Variant}
	\begin{itemize}
		\item Sometimes the proposition to be proven is {\bf not} of the form  $A \Rightarrow B$
		\item Instead only a proposition $S$ is to be proven
		\item This is a short form of $K \Rightarrow S$, where $K$ is the already proven knowledge or the axioms about the underlying universe of discourse
		\item In this case it is to proof $K \land \lnot S \Rightarrow \text{{\bf F}}$
		\item In short: In order to proof $S$ we assume $\lnot S$ and show that this leads to ``absurdity''
	\end{itemize}
\end{frame}
\begin{frame}
	\frametitle{Proof by Contradiction}
	\framesubtitle{A Specific Variant}
	\begin{Example}
		\begin{itemize}
			\item $\lnot (\exists n \in \mathbb{N}, \forall i \in \mathbb{N}: i \leq n)$ aka ``There is no greatest natural number''
			\item We assume $\exists n \in \mathbb{N}, \forall i \in \mathbb{N}: i \leq n$ aka "There is a greatest natural number"
			\item Lets take a new natural number $k = n + 1$ which exists since $n \in \mathbb{N}$ and adding 1 to $n$ is again a natural number (since $\mathbb{N}$ is closed under +).
			\item Now we have a new natural number $k > n$ which contradicts our assumption
			\item Therefore the initial proposition is proven
		\end{itemize}
	\end{Example}

\end{frame}
\end{document}
