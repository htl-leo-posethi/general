\documentclass[a4paper,11pt]{exam}

\pointpoints{\%}{\%}

\usepackage{amsthm}
\usepackage{amsmath}
\usepackage{amsfonts}
\usepackage{amssymb}
\usepackage[german, english]{babel}
\usepackage[utf8]{inputenc}

\theoremstyle{definition}
\newtheorem{remark}{Remark}
\begin{document}

\printanswers

\begin{center} \fbox{\fbox{\parbox{5.5in}{\centering
IF.05.22 --- Theoretical Informatics --- Proof Exercises}}}
\end{center}
%\makebox[\textwidth]{Name: \enspace\hrulefill}


\newtheorem{theorem}{Theorem}
\theoremstyle{definition}
\newtheorem{example}{Example}

\begin{questions}
\question{\bf Proof by Contradiction}
Let $a, b \in \mathbb{Z}$. Proof the following proposition:
\[a + b \geq 19 \Rightarrow a \geq 10 \lor b \geq 10.\]
Take care to show every step of the proof carefully. How does such a proof look like, what do you assume, what do you have to show, where is the contradiction?

\begin{otherlanguage}{german}
Seien $a, b \in \mathbb{Z}$. Beweisen Sie folgende Aussage:
\[a + b \geq 19 \Rightarrow a \geq 10 \lor b \geq 10.\]
Beachten Sie, dass Sie jeden Beweisschritt sorgfältig protokollieren. Wie sieht so ein Beweis aus, was nehmen Sie an, was zeigen Sie, wo ist der Widerspruch?
\end{otherlanguage}

\question{\bf Indirect Proof}
Let $a, b, c \in \mathbb{R}^+$. Proof the following proposition via an indirect proof:
\[a \cdot b = c \Rightarrow a \leq \sqrt{c} \lor b \leq \sqrt{c}.\]
Take care to show every step of the proof carefully. In particular accomplish the following tasks:
\begin{parts}
	\part Write down the general scheme of an indirect proof.
	\part Write down the starting point to proof the above proposition indirectly.
	\part Transform the starting point of your proof such that it becomes obvious in which range the required values for $a$ and $b$ are to be found.
	\part Give values of $a$ and $b$ to be able to finalize the proof.
	\part Finalize the proof.
	\part Take care of a clear and mathematically oriented structure of your proof.
\end{parts}
{\bf Hint:} If you need to express a real number $x > a$ you may assume an infinitely small number $\varepsilon > 0$ to construct it.

\begin{otherlanguage}{german}
Seien $a, b, c \in \mathbb{R}^+$. Beweisen Sie folgende Aussage mit Hilfe eines indirekten Beweises:
\[a \cdot b = c \Rightarrow a \leq \sqrt{c} \lor b \leq \sqrt{c}.\]
\begin{parts}
	\part Geben Sie die allgemeine Form eines indirekten Beweises an.
	\part Geben Sie den Ansatz für einen indirekten Beweis der obigen Behauptung.
	\part Formen Sie den Ansatz so um, dass offensichtlich wird, in welchem Bereich Werte für $a$ und $b$ gesucht sind.
	\part Bestimmen Sie Werte für $a$ and $b$, mit denen Sie den Beweis fertigstellen können.
	\part Stellen Sie den Beweis fertig.
	\part Achten Sie auf eine klare, mathematikorientierte Struktur Ihres Beweises.
\end{parts}
{\bf Hint:} Falls Sie eine reelle Zahl $x > a$ für Ihren Beweis brauchen, dürfen Sie annehmen, dass es eine unendlich kleine Zahl $\varepsilon > 0$ mit der Sie $x$ dann konstruieren können.
\end{otherlanguage}
\begin{solution}
	\begin{parts}
		\part $A \Rightarrow B \equiv \lnot B \Rightarrow \lnot A$
		\part $\lnot (a \leq \sqrt{c} \lor b \leq \sqrt{c}) \Rightarrow a \cdot b \neq c$
		\part Using DeMorgan we can rewrite the above formula to ($a > \sqrt{c} \land b > \sqrt{c}) \Rightarrow a \cdot b \neq c$
		\part We are looking for the smallest number $a$ and $b$ both $> \sqrt{c}$. So we assume a number $\varepsilon > 0$ and let $a, b = \sqrt{c + \varepsilon}$.
		\part In order to proof that $a \cdot b \neq c$ under the above condition it is sufficient to show that $\sqrt{c + \varepsilon} \cdot \sqrt{c + \varepsilon} \neq c$.
		
		$\sqrt{c + \varepsilon} \cdot \sqrt{c + \varepsilon} = (\sqrt{c + \varepsilon})^2 = c + \varepsilon$ which is obviously unequal to $c$.
	\end{parts}
\end{solution}

\question{\bf Proof by Contradiction}
Proof that there are infinitely many odd numbers. Hint: For the sake of the proof assume that there are finitely many odd numbers. Now add all these numbers. Is the sum of these numbers odd or even? If it is even, how can you gain an odd number from this?

\end{questions}

\end{document}