\documentclass{beamer}
%%% BEGIN PREAMBLE

\usepackage{amsmath}
\usepackage{amsfonts}
% \usepackage{beamerthemesplit} // Activate for custom appearance
\usepackage{graphicx}
\usepackage[utf8]{inputenc}

% Definitions
\theoremstyle{remark}
\newtheorem{remark}[theorem]{Remark}

% Title stuff
\title{Unit 02 -- Propositional Logic -- Basics}
\author{Peter Bauer}
\date{} % delete this line to display the current date

%%% BEGIN DOCUMENT
\begin{document}

\frame{\titlepage}

\begin{frame}
	\frametitle{Outline}
	\tableofcontents
\end{frame}

\section{Propositions and Propositional Variable}
\begin{frame}
\frametitle{Proposition}
\begin{itemize}
	\item Propositions are done in everyday life.
	\pause
	\item For propositions it makes sense to ask whether they are ``true'' or ``false''.
	\pause
	\item The answer to this question is either ``true'' or ``false'' (nothing in between).
	\pause
	\item Example: ``The circuit is electrically conducting.''
\end{itemize}	
	
\end{frame}

\begin{frame}
\frametitle{Propositional Variable}
\begin{itemize}
	\item Instead of propositions we are often talking of {\em propositional variables}
	\pause
	\item Expresses that we are not interested in a specific proposition but a variable that can hold arbitrary propositions
	\pause
	\item Simplified we can understand propositional variables as variables that hold either the value $t$ or $f$
	\pause
	\item Instead of ``Let $A$ be an arbitrary proposition'' we can say ``Let $A$ be a propositional variable''
\end{itemize}
\end{frame}

\begin{frame}
\frametitle{Proposition --- Definition}
\begin{definition}
A {\em proposition} is a ``linguistic item'' for which it makes sense to ask whether it is ``true'' or ``false''. The terms ``true'' and ``false'' are called the {\em truth values} of a proposition.
\end{definition}

\pause
\begin{remark}
\begin{itemize}
 	\item Propositions are mainly denoted by upper case letters $A$, $B$, $C$, \ldots
	\pause
	\item  ``true'' is often abbreviated by $t$ and ``false'' by $f$
	\pause
	\item We are only interested in the structure of propositions and not in its content.
\end{itemize}

\end{remark}
\end{frame}

\begin{frame}
 \frametitle{Examples}
 
\begin{itemize}
	\item $x \leq 3$
	\pause
	\item In German: Wenn Anastasia den Most holt, wird es bald etwas zu trinken geben, und das Abendessen wird beginnen, vorausgesetzt, dass Bartholomäus das Brot schon gebacken hat.
\end{itemize}
\end{frame}


\section{Compositions of Propositions}
\begin{frame}
\frametitle{Basic Compositions}

\begin{definition}
Let $A$ and $B$ be two propositions. Then the truth values of the propositions $A \land B$ (``$A$ and $B$''), $A \lor B$ (``$A$ or $B$), and $\lnot A$ (``not $A$'') are given by the following truth table:


\begin{tabular}{c|c||c|c|c}
$A$ & $B$ & $A \land B$ & $A \lor B$ & $\lnot A$ \\ \hline \hline
$t$ & $t$ & \pause $t$ & $t$ & $f$ \\ \pause
$t$ & $f$ &  \pause $f$ & $t$ & $f$ \\ \pause
$f$ & $t$ &  \pause $f$ & $t$ & $t$ \\ \pause
$f$ & $f$ &  \pause $f$ & $f$ & $t$
\end{tabular}
\end{definition}

\pause
\begin{remark}
	$\land$ is called {\em conjunction}, $\lor$ is called {\em disjunction} and $\lnot$ is called {\em negation}.
\end{remark}

\end{frame}

\begin{frame}
\frametitle{Implication}
\begin{definition}
Let $A$ and $B$ be two propositions. Then the {\em implication} $A \Rightarrow B$ (``if $A$ then $B$'') is defined by the following truth table:
\pause
\begin{tabular}{c|c||c}
$A$ & $B$ & $A \Rightarrow B$ \\ \hline \hline
$t$ & $t$  \pause & $t$  \\
$t$ & $f$ \pause & $f$  \\
$f$ & $t$ \pause & $t$ \\
$f$ & $f$ \pause & $t$ 
\end{tabular}

\end{definition}
\end{frame}

\begin{frame}
\frametitle{Equivalence}
\begin{definition}
Let $A$ and $B$ be two propositions. Then the {\em equivalence} $A \Leftrightarrow B$ (``$A$ if and only if  $B$'') is defined by the following truth table:
\pause
\begin{tabular}{c|c||c}
$A$ & $B$ & $A \Leftrightarrow B$ \\ \hline \hline
$t$ & $t$ \pause & $t$  \\
$t$ & $f$ \pause & $f$  \\
$f$ & $t$ \pause & $f$ \\
$f$ & $f$ \pause & $t$ 
\end{tabular}

\end{definition}
\end{frame}

\begin{frame}
 \frametitle{Remarks}
 
\begin{itemize}
	\item The composition of two propositions is again a proposition
	\pause
	\item The compositions are binary, i.e., they have two ``operands''
	\pause
	\item By nesting compositions and using parentheses we can construct more complex propositions (more specific propositional formulas)
	\pause
	\item To reduce the number of necessary parentheses we use the precedence rules $\lnot$ has higher precedence than $\land$ higher than $\lor$ higher than $\Rightarrow$ higher than $\Leftrightarrow$
	\pause
	\item The syntax of the compositions has to be considered, e.g., $\lnot \land AB$ is NO proposition
\end{itemize}
\end{frame}

\section{Propositional Formulas}
\begin{frame}
\frametitle{Propositional Formula}
\begin{definition}
Given a set of propositional variables or propositions $A, B, C, \ldots$. If we combine these propositions by means of the compositions given above, the result of this compositions is called a {\em propositional formula}. If we assign true or false propositions to each of the propositional variables in a propositional formula we get a concrete proposition.
\end{definition}
\end{frame}

\begin{frame}
\frametitle{Equivalence and Implication of Propositional Formulas}
\begin{definition}
Let $P = P(A, B, C, \ldots)$ and $Q = Q(A, B, C, \ldots)$ be propositional formulas then
\begin{enumerate}
	\item $P$ and $Q$ are called {\em equivalent} ($P \equiv Q$), if $P$ and $Q$ have the same truth value for each possible assignment of their propositional variables
	\pause
	\item $P$ {\em implies} $Q$ ($P \vdash Q$) if for each assignment to the propositional variables holds that if $P$ is true then also $Q$ is true.
\end{enumerate}
\end{definition}
\pause
\begin{remark}
Very often we loosely interchange the symbols $\equiv$ and $\Leftrightarrow$ as well as $\vdash$ and $\Rightarrow$
\end{remark}
\end{frame}

\section{Properties of Compositions}
\begin{frame}
\frametitle{Commutativity, Associativity}
\begin{theorem}
\begin{enumerate}
	\item Commutativity	\begin{eqnarray*}
						A \land B &\equiv &B \land A \\
						A \lor B &\equiv & B \lor A
					\end{eqnarray*}
	\pause
	\item Associativity	\begin{eqnarray*}
						A \land (B \land C) & \equiv & (A \land B) \land C \equiv A \land B \land C \\
						A \lor (B \lor C) & \equiv & (A \lor B) \lor C \equiv A \lor B \lor C 
					\end{eqnarray*}
\end{enumerate}
\end{theorem}
\end{frame}

\begin{frame}
\frametitle{Distributivity, De-Morgan}
\begin{theorem}
\begin{enumerate}
	\item Distributivity	\begin{eqnarray*}
						A \land (B \lor C) & \equiv & (A \land B) \lor (A \land C) \\
						A \lor (B \land C) & \equiv & (A \lor B) \land (A \lor C) \\
					\end{eqnarray*}
	\pause
	\item De-Morgan	\begin{eqnarray*}
						\lnot(A \land B) & \equiv & \lnot A \lor \lnot B \\
						\lnot(A \lor B) & \equiv & \lnot A \land \lnot B
					\end{eqnarray*}
\end{enumerate}
\end{theorem}
\end{frame}

\begin{frame}
\frametitle{Absorption, Idempotence, and Double Negation}
\begin{theorem}
\begin{enumerate}
	\item Absorption	\begin{eqnarray*}
						A \land (A \lor B) & \equiv &  A \\
						A \lor (A \land B) & \equiv &  A 
					\end{eqnarray*}
	\pause
	\item Idempotence	\begin{eqnarray*}
						A \land A & \equiv & A \\
						A \lor A & \equiv & A \\
					\end{eqnarray*}
	\pause
	\item Double Negation \begin{eqnarray*}
		\lnot(\lnot A) & \equiv & A 
	\end{eqnarray*}
\end{enumerate}
\end{theorem}
\end{frame}

\begin{frame}
 \frametitle{Tautology and Contradiction}
 \begin{definition}
 A formula which is always ``true'', no matter which truth values are assigned to its variables is called a {\em tautology} (sign {\bf T}). A formula which is always ``false'', no matter which truth values are assigned to its variables is called a {\em contradiction} (sign {\bf F}).
 \end{definition}
 \pause
\begin{example}
 It is easy to show that $(A \Rightarrow B) \Leftrightarrow (\lnot A \lor B)$ is a tautology.
\end{example}
\end{frame}

\begin{frame}
 \frametitle{Laws with Contradictions and Tautologies}
 
\begin{theorem}
 	For a proposition $A$ holds:
	
\begin{eqnarray*}
 A \land \mathbf{T}  \equiv A, &  A \lor \mathbf{T} \equiv \mathbf{T} \\
 A \land \mathbf{F}  \equiv \mathbf{F}, &  A \lor \mathbf{F} \equiv A \\
 A \land \lnot A \equiv \mathbf{F}, & A \lor \lnot A \equiv \mathbf{T}
\end{eqnarray*}
\end{theorem}
\end{frame}

\end{document}
