\documentclass{beamer}
%%% BEGIN PREAMBLE

\usepackage{amsthm}
\usepackage{tikz}
\usetikzlibrary{automata, arrows.meta, positioning}
\usepackage{listings}
% \usepackage{beamerthemesplit} // Activate for custom appearance

% Definitions
\theoremstyle{remark}
\newtheorem{remark}[theorem]{Remark}

\newcommand{\emptystring}{\varepsilon}
% \newcommand{\length}[1]{\mid #1 \mid}
% \newcommand{\directlygenerates}{\Rightarrow}
% \newcommand{\generates}{\Rightarrow^{+}}
% \newcommand{\generatesequal}{\Rightarrow^{*}}

\newenvironment{grammar}
	{\begin{tabular}[b]{lcl}}
	{\end{tabular}}
\newcommand{\rewritten}{$\to$}
\newcommand{\alternative}{$\mid$\;}

% Title stuff
\title{Regular Languages}
\author{Peter Bauer}
\date{V\_01.03.01}

%%% BEGIN DOCUMENT
\begin{document}

\frame{\titlepage}

\begin{frame}
	\frametitle{Outline}
	\tableofcontents
\end{frame}

\section {Motivation}
\begin{frame}
	\frametitle{Motivation}
	
	\begin{itemize}
		\item Simplest type of grammar
		\item Applications in pattern matching
		\item Implementations
		\begin{itemize}
			\item lex / flex
			\item In editors for searching (vi, emacs, ...)
			\item In many scripting languages (Perl, Python, JavaScript, ...)
			\item In frameworks like .NET ({\tt System.Text.RegularExpressions}) or Java ({\tt java.util.regex})
		\end{itemize}
	\end{itemize}
\end{frame}

\section{Definitions}
\subsection {Regular Grammar}
\begin{frame}
	\frametitle{Regular Grammar}
	\begin{definition}
		A grammar $G$ is called a {\em regular grammar} if it has only rules of the form
		\begin{itemize}
			\item $A$ \rewritten $\alpha$,
			\item $A$ \rewritten $\alpha B$,
		\end{itemize}
		where $A, B \in V_N$ and $\alpha \in V_T$.
	\end{definition}
	
	\pause
	
	\begin{remark}
		$G$ may also have the rule $S$ \rewritten $\varepsilon$ if $S$ does not appear on the right side of the grammar.
	\end{remark}
\end{frame}

\begin{frame}
	\frametitle{Reverse Regular Grammar}
	\begin{definition}
		A grammar is called a {\em reverse regular grammar} if it has only rules of the form $A$ \rewritten $\alpha$ or
		$A$ \rewritten $B \alpha$, where $A, B \in V_N$ and $\alpha \in V_T$.
	\end{definition}
	
	\pause
	
	\begin{remark}
		$G$ may also have the rule $S$ \rewritten $\varepsilon$ if $S$ does not appear on the right side of the grammar.
	\end{remark}
\end{frame}

\begin{frame}
	\frametitle{An Example}
	\begin{example}
		The following regular grammar describes a simple form of an identifier. $c$ is a terminal class
		of characters and $d$ is a terminal class of digits.
		
		\begin{grammar}
			$S$ & \rewritten & $c$ \alternative $cR$\\
			$R$ & \rewritten & $c$ \alternative $d$ \alternative $cR$ \alternative $dR$
		\end{grammar}
	\end{example}
\end{frame}

\subsection{Regular Language}
\begin{frame}
	\frametitle{Regular Language}
	\begin{definition}
		The language $L(G(S))$ of a regular grammar $G$ is a {\em regular language} over an alphabet $\Sigma$ and is defined as follows:
		\begin{itemize}
			\pause
			\item The empty set $\Phi$ is a regular language.
			\pause
			\item The set with the empty string $\{\emptystring \}$ is a regular language.
			\pause
			\item For each $\omega \in \Sigma$ the language $\{\omega\}$ is a regular language.
			\pause
			\item If $A$ and $B$ are regular languages, then $A \cup B$, $AB$ (concatenation), and $A^*$ (Kleene star) are
			regular languages.
			\pause
			\item No other language over $\Sigma$ is a regular language.
		\end{itemize}
	\end{definition}
\end{frame}

\subsection{Regular Set}
\begin{frame}
	\frametitle{Regular Set}

	\begin{definition}
		Analogously to regular languages we can define {\em regular sets} as follows:
		\begin{itemize}
			\item $\Phi$,  $\{\emptystring \}$, and $\{\omega\}$ for each $\omega \in \Sigma$ are regular sets.
			\item If $A$ and $B$ are regular sets, then $A \cup B = \{\alpha \mid \alpha \in A \lor \alpha \in B\}$,
			$AB = \{\alpha \beta \mid \alpha \in A \land \beta \in B\}$,
			and $A^* = \underset{n \ge 0}{\bigcup}P_n$ are regular sets.
			\item Nothing else is a regular set.
		\end{itemize}
	\end{definition}

	\pause
	
	\begin{example}
	Let $V = \{\mathtt{c}, \mathtt{d}\}$ be a set where $\mathtt{c}$ stands for a character and $\mathtt{d}$ stands for a digit.
	$\{c\} (\{c\} \cup \{d\})^*$ is a regular set that describes a simple form of identifier.
	\end{example}
\end{frame}

\subsection{Regular Expression}
\begin{frame}
	\frametitle{Regular Expression}
	Regular expression describe regular sets by
	\begin{itemize}
		\item Omitting the curled brackets
		\item Writing $+$ instead of $\cup$
	\end{itemize}
	
	\pause
	
	\begin{remark}
		\begin{itemize}
			\item To avoid parentheses the following priority is assumed:
			\begin{enumerate}
				\item Kleene star has the highest priority
				\item Concatenation has the second highest priority
				\item $+$ has the lowest priority 
			\end{enumerate}
			
			\item Sometimes $\mid$ is used instead of $+$
		\end{itemize}
	\end{remark}
	
	\pause
	
	\begin{example}
		The identifier example from above would boil down to $\mathtt{c}(\mathtt{c}+\mathtt{d})^*$.
	\end{example}
\end{frame}

\section{Finite Automata}
\subsection{Deterministic Finite Automaton (DFA)}
\begin{frame}
	\frametitle{Deterministic Finite Automaton -- DFA}
	\begin {definition}
	A {\em Deterministic Finite Automaton} (DFA) is a quintuple $(Q, \Sigma, \delta, q_0, F)$ where
	\begin {itemize}
		\item $Q$ is a finite set of states
		\item $\Sigma$ is a set of inputs
		\item $\delta: Q \times \Sigma \to Q$ the state transition function
		\item $q_0$ the start state
		\item $F \subseteq Q$ the final states
	\end{itemize}
	\end{definition}
	
	\pause
	
	\begin{remark}
		DFAs are often depicted by means of state diagrams
	\end{remark}
\end{frame}

\begin{frame}
	\frametitle{State Diagrams}
	\begin{itemize}
		\item $Q = \{A, B\}$
		\item $\Sigma = \{0, 1\}$
		\item $\delta(A, 1) = B, \delta(B, 0) = B$
		\item $q_0 = A$
		\item $F = \{B\}$
	\end{itemize}
	
	\pause
	
	\begin{center}
		\begin{tikzpicture}[node distance = 3cm, on grid, auto]
		%\draw[help lines] (-1,1) grid (3,-1);
		\node (A)[state, initial, initial text=] {A};
		\node (B)[state, right = of A, accepting] {B};
		\path [-stealth, thick]
			(A) edge node {1} (B)
			(B) edge [loop above] node {0} ();
	\end{tikzpicture}
\end{center}
		
	\pause
	
	\begin{itemize}
		\item States are circles
		\item Transitions are arrows
		\item Inputs are labels on the arrows
		\item Initial states are circles with an arrow from "nowhere"
		\item Final states are double circles
	\end{itemize}
\end{frame}

\begin{frame}
	\frametitle{Coffee Machine}
	\begin{itemize}
		\item Coins of 10, 20 and 50 cents are accepted
		\item Machine does NOT give change, i.e., if a coin is inserted which would cause an ``overpay'' the coin is thrown into the return tray and the machine doesn't change state.
		\item After a total amount of 50 cents inserted ``c'' for coffee and ``t'' for tea can be chosen.
		\item When the machine prepares tea or coffee it is in a state T or C, respectively.
		\item When the chosen drink is ready the user must press a button ``o'' to open the chamber in order to grab the cup and the machine goes back into its initial state.
	\end{itemize}
\end{frame}

\begin{frame}
	\frametitle{DFA of a Coffee Machine}
	\begin{center}
		\begin{tikzpicture}[node distance = 1.8cm, on grid, auto]
			\node (R)[state, initial, initial text=, accepting] {R};
			\node (10)[state, right = of R] {10};
			\node (20)[state, below right = of 10] {20};
			\node (30)[state, above right = of 10] {30};
			\node (40)[state, above right = of 20] {40};
			\node (50)[state, below right = of 40] {50};
			\node (C)[state, below = of 20, yshift=-.4cm] {C};
			\node (T)[state, below = of C, yshift=.5cm] {T};

			\path[-stealth, thick]
				(R) edge node {10} (10)
				(R) edge [bend right=40] node [below] {20} (20)
				(R) edge [bend right=65] node [below] {50} (50)
				(10) edge node {10} (20)
				(10) edge node {20} (30)
				(10) edge [loop below] node {50} ()
				(20) edge node {20} (40)
				(20) edge [loop below] node [right]{50} ()
				(30) edge [bend right] node {10} (40)
				(30) edge [loop above] node {50} ()
				(30) edge [bend left] node {20} (50)
				(40) edge [bend right] node {10} (50)
				(40) edge [loop below] node {20, 50} ()
				(50) edge [loop right] node {10, 20, 50} ()
				(50) edge [bend left] node {c} (C)
				(50) edge [bend left] node {t} (T)
				(C) edge [bend left] node {r} (R)
				(T) edge [bend left] node {r} (R);
		\end{tikzpicture}
	\end{center}
\end{frame}

\begin{frame}
	\frametitle{DFA Accepting a Simple Identifier}
	
	\begin{grammar}
		$S$ & \rewritten & $c$ \alternative $cR$\\
		$R$ & \rewritten & $c$ \alternative $d$ \alternative $cR$ \alternative $dR$
	\end{grammar}
	
	\pause

	\begin{center}
		\begin{tikzpicture}[node distance = 3cm, on grid, auto]
			\node (S) [state, initial, initial text =] {S};
			\node (R) [state, accepting, right = of S] {R};

			\path[-stealth, thick]
				(S) edge node {c} (R)
				(R) edge [loop] node [above] {c, d} ();

		\end{tikzpicture}			
	\end{center}
\end{frame}

\begin{frame}[fragile]
	\frametitle{DFA Syntax Check Procedure}

	\begin{itemize}
		\item \lstinline$checkString$ is a function of a class \lstinline|DFA|
		\item \lstinline|input| is the string to be checked
		\item \lstinline|delta| is the state transition function $\delta$
	\end{itemize}

	\lstset{
		basicstyle=\scriptsize,
		keywordstyle=\color{blue},
		commentstyle=\color{teal},
		tabsize=2
	}
	\begin{lstlisting}[language=Swift]
func checkString(_ input: String) -> Bool {
	var currentState: State = initialState
	var stringIsValid = true
	for c in input {
		if let nextState = delta(currentState, c), stringIsValid {
			// transition exists and we are still on track
			currentState = nextState
		} else {
			stringIsValid = false
		}
	}
	return stringIsValid && finalStates.contains(currentState)
}
		\end{lstlisting}	
\end{frame}

\subsection{Transformations}
\begin{frame}
	\frametitle{Transformation of a DFA to a Regular Expression --- Step 1}
	Replace different transitions from one state to another into one 
	
	\begin{center}
		\begin{tikzpicture}[->, >=stealth, node distance=3cm, on grid, auto]
			\node[state] (qi) {$q_i$};
			\node[state, right of=qi] (qj) {$q_j$};
		
			\path (qi) edge[bend left] node[above] {$a$} (qj)
					edge[bend right] node[below] {$b$} (qj);
		\end{tikzpicture}
	  
		\vspace{2em}

		\begin{tikzpicture}[->, >=stealth, node distance=3cm, on grid, auto]
			\node[state] (qi) {$q_i$};
			\node[state, right of=qi] (qj) {$q_j$};
		  
			\path (qi) edge[bend left] node[above] {$a + b$} (qj);
		\end{tikzpicture}	
	\end{center}
\end{frame}

\begin{frame}
	\frametitle{Transformation of a DFA to a Regular Expression --- Step 2}
	Remove intermediate states 
	
	\begin{center}
		\begin{tikzpicture}[->, >=stealth, node distance=2.5cm]
			\node[state] (qi) {$q_i$};
			\node[state, right of=qi] (q) {$q$};
			\node[state, right of=q] (qj) {$q_j$};
		
			\path[draw=red]
				(qi) edge[bend left] node[above] {$a$} (q)
				(q) edge[loop above] node {$e$} (q)
				(q) edge[bend left] node[above] {$b$} (qj);
			\path[draw=blue]
				(qj) edge[bend left] node[below] {$c$} (q)
				(q) edge[bend left=50] node[below] {$d$} (qi);
		\end{tikzpicture}
			
		\vspace{2em}

		\begin{tikzpicture}[->, >=stealth, node distance=5cm]
			\node[state] (qi) {$q_i$};
			\node[state, right of=qi] (qj) {$q_j$};
		
			\path[draw=red]
				(qi) edge[bend left] node[above] {$ae^*b$} (qj);
			\path
				(qj) edge[bend left] node[below] {$ce^*d$} (qi)
				(qi) edge[loop above] node {$ae^*d$} (qi)
				(qj) edge[loop above] node {$ce^*b$} (qj);
		\end{tikzpicture}
	\end{center}

\end{frame}

\begin{frame}
	\frametitle{Transformation of a DFA to a Regular Expression --- Step 3}
	Remove remaining states 
	
	\begin{center}
		\begin{tikzpicture}[->, >=stealth, node distance=5cm]
			\node [state] (qi) {$q_i$};
			\node[state, right of=qi] (qj) {$q_j$};
		
			\path (qi) edge[bend left] node[above] {$r_2$} (qj)
				(qj) edge[bend left] node[below] {$r_3$} (qi)
				(qi) edge[loop above] node {$r_1$} (qi)
				(qj) edge[loop above] node {$r_4$} (qj);
		\end{tikzpicture}
	\end{center}

	\begin{itemize}
		\item We need $r_1^*r_2$ in any way to go from $q_i$ to $q_j$.
		\item Then we have two possibilities to reach $q_j$ again:
		\begin{itemize}
			\item Either we repeat $r_4$
			\item Or we repeat $r_3r_1^*r_2$
		\end{itemize}
		\item This results in $r_1^*r_2(r_4 + r_3r_1^*r_2)^*$
	\end{itemize}
\end{frame}

\begin{frame}
	\frametitle{Transformation of a Regular Expression to a DFA}
	Given a regular expression $\mathtt{a}$ \alternative $\mathtt{b}$($\mathtt{cc}$* \alternative ($\mathtt{a}$ \alternative $\mathtt{c}$)*)$\mathtt{b}$
		\begin{enumerate}
		\item Subscript each symbol of the regular expression with a unique index:
		$\mathtt{a}_1$ \alternative $\mathtt{b}_2$($\mathtt{c}_3\mathtt{c}_4$* \alternative ($\mathtt{a}_5$  \alternative $\mathtt{c}_6$)*)$\mathtt{b}_7$.
		
		The indices are the states of the DFA.
		
		\pause
		
		\item Create an initial state by adding a state 0 at the very beginning of the expression:
		$_0$($\mathtt{a}_1$ \alternative $\mathtt{b}_2$($\mathtt{c}_3\mathtt{c}_4$* \alternative ($\mathtt{a}_5$  \alternative $\mathtt{c}_6$)*)$\mathtt{b}_7$).
		
		\item Determine the final states by looking for states which don't require to have a symbol followed. In our example these are the states
		\pause
		1 and 7.
		
		\pause
		
		\item Construct a transition matrix by documenting for each state, which symbols may follow and into which subsequent states the automaton will pass. See next slide
	\end{enumerate}

\end{frame}

\begin{frame}
	\frametitle{Example Continued}
	$_0$($\mathtt{a}_1$ \alternative $\mathtt{b}_2$($\mathtt{c}_3\mathtt{c}_4$* \alternative ($\mathtt{a}_5$  \alternative $\mathtt{c}_6$)*)$\mathtt{b}_7$).
	
	\begin{enumerate}
		\item The first transition from 0 is to 1 by reading an {\tt a}. The second transition from 0 is to 2 by reading a {\tt b}. 1 is a final state.
		
		\pause
		
		\item The transitions from 2 are to the states 3 (c), 5 (a), 6 (c), and 7 (b).
		
		\pause
		
		\item The transitions from 3 are to the states 4 (c) and 7 (b).
		
		\pause
		
		\item The transitions from 4 are to 4 (c) and to 7(b).
		
		\pause
		
		\item The transitions from 5 are to 5 (a), to 6 (c), and to 7 (b).
		
		\pause
		
		\item The transitions from 6 are to 6 (c) to 5 (a), and to 7 (b).
		
		\pause
		
		\item 7 is a final state.
	\end{enumerate}
\end{frame}
\end{document}
