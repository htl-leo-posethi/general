\documentclass{beamer}
%%% BEGIN PREAMBLE

\usepackage{amsmath}
\usepackage{amsfonts}
% \usepackage{beamerthemesplit} // Activate for custom appearance
\usepackage{graphicx}
\usepackage[utf8]{inputenc}

% Definitions
\theoremstyle{remark}
\newtheorem{remark}[theorem]{Remark}

% Title stuff
\title{Unit 03 – Minimizing Formulas}
\author{Peter Bauer}
\date{} % delete this line to display the current date

%%% BEGIN DOCUMENT
\begin{document}

\frame{\titlepage}

\begin{frame}
	\frametitle{Outline}
	\tableofcontents
\end{frame}

\section{Propositional Logic Proofs}
\begin{frame}
	\frametitle{Minimizing and Proofs}
	\framesubtitle{The similarities}
	\begin{itemize}
		\item Given: a propositional formula $P$
		\item Wanted: a propositional formula $Q$
		\item where $P \equiv Q$
	\end{itemize}
\end{frame}

\begin{frame}
	\frametitle{Minimizing and Proofs}
	\framesubtitle{The differences}
	\begin{itemize}
		\item When minimizing
		\begin{itemize}
			\item $Q$ should me ``minimal''
			\item e.g., by number of variables, length of formula, \ldots
		\end{itemize}
		\item When proofing
		\begin{itemize}
			\item $Q$ is also given
			\item We want to show that we can ``go'' from $P$ to $Q$
			\item Using small steps which are already proven (aka Properties of Compositions)
		\end{itemize}
	\end{itemize}
\end{frame}
\AtBeginEnvironment{align}{\setcounter{equation}{0}}
\begin{frame}
	\frametitle{A First Example}
	We want to proof that $C \lor \lnot(B \land C) \equiv \text{{\bf T}}$
	\pause
	\begin{align}
		C \lor \lnot(B \land C)  & \equiv & \text{[DeMorgan]} \\
		C \lor (\lnot B \lor \lnot C) & \equiv & \text{[Commut.]} \\
		C \lor (\lnot C \lor \lnot B) & \equiv & \text{[Assoc.]} \\
		(C \lor \lnot C) \lor \lnot B & \equiv & \text{[Taut.]} \\
		\text{{\bf T}} \lor \lnot B & \equiv & \text{[Taut.]} \\
		\text{{\bf T}} & \hspace{1em}\square
	\end{align}
\end{frame}

\begin{frame}
	\frametitle{A Second Example}
	\framesubtitle{Your Turn}
	Proof that $(P \lor \lnot Q) \Rightarrow Q \equiv (\lnot P \lor Q) \land Q$

	Bring the lines below in the right order and write down the properties used.

	{\bf Hint:} Here a property $A \Rightarrow B \equiv \lnot A \lor B$ is used. We call it the Implication Elimination Rule (IER).

	\begin{align}
		(P \lor \lnot Q) \Rightarrow Q & \equiv \\
		(\lnot P \lor Q) \land Q & \hspace{1em}\square \\
		(\lnot P \lor Q) \land (Q \lor Q) & \equiv \\
		(\lnot P \land Q) \lor Q & \equiv \\
		\lnot(P \lor \lnot Q) \lor Q & \equiv
	\end{align}
\end{frame}

\begin{frame}
	\frametitle{A Third Example}
	\framesubtitle{Again Your Turn}
	Proof that $A \Rightarrow B \equiv A \land \lnot B \Rightarrow \text{{\bf F}}$

	\newcommand{\ph}{\underline{\hspace{1em}}\hspace{.5em}}

	\begin{align}
		A \Rightarrow B & \equiv & [IER] \\
		\lnot A \hspace{.5em} \ph \ph & \equiv & [DeMorgan] \\
		\ph(\lnot\lnot\ph\ph\ph\ph) & \equiv & [Double Negation] \\
		\ph(\ph \ph \ph \ph) & \equiv & [Tautology]\\
		\ph(\ph \ph \ph \ph) \ph \text{{\bf F}} & \equiv & [IER] \\
		\ph \ph \ph \ph \Rightarrow \ph & \hspace{1em}\square
	\end{align}

\end{frame}

\section{Disjunctive Normal Form}
\begin{frame}
	\frametitle{Disjunctive Normal Form -- Motivation}
	\begin{itemize}
	\item Given the truth table of a propositional formula $P$
	\item Find a logically equivalent formula $P^{\prime}$
	\pause
	\item No unambiguous solution
	\pause
	\item Disjunctive normal form is one (and easy to find) solution
	\pause
	\item Is a basis for several methods of several algorithms for automated formula simplification
\end{itemize}

\end{frame}

\begin{frame}
	\frametitle{Disjunctive Normal Form}
	\begin{definition}
	A propositional logic formula of the form
	\begin{align*}
		(A_{11} \land A_{12} \land \ldots \land A_{1n}) & \lor \\
		(A_{21} \land A_{22} \land \ldots \land A_{2n}) & \lor \\
		\ldots \\
		(A_{m1} \land A_{m2} \land \ldots \land A_{mn}) & \\
	\end{align*}
	is called a {\em formula in disjunctive normal form} (DNF).
\end{definition}

\end{frame}

\begin{frame}
\frametitle{Procedure to Find the DNF}
\begin{itemize}
	\item Given a propositional formula $P(A_1, A_2, \ldots A_n)$ with $n$ propositional variables
	\pause
	\item Calculate the truth table of $P$
	
	\item For each row $j$ in the truth table, where $P=t$ model a sub-formula $Q_j$ as follows:
	\pause
	\[Q_j = \left ( \left ( \begin{matrix}A_1 \\ \text{or} \\ \lnot A_1  \end{matrix} \right ) \land
		\left ( \begin{matrix}A_2 \\ \text{or} \\ \lnot A_2  \end{matrix} \right ) \land \ldots \land
		\left (\begin{matrix}A_n \\ \text{or} \\ \lnot A_n  \end{matrix}\right ) \right ). \]
	
	\item Choose $A_i$ for the $i^{\text{th}}$ variable if the truth table holds a $t$ and $\lnot A_i$ otherwise.
	\pause
	\item The complete formula $P^{\prime}$ is realized by assembling all the sub-formulas $Q_1, Q_2, \ldots, Q_m$ as follows:
		\[P^{\prime} = Q_1 \lor Q_2 \lor \ldots \lor Q_m\]
\end{itemize}
\end{frame}

\begin{frame}
\frametitle{Example}

\begin{example}
	Consider the formula $P(A, B) = A \Rightarrow B$. The truth table would be as follows:
	\begin{center}
	\begin{tabular}{c|c||c||c}
		$A$ & $B$ & $P(A, B)$ & $Q_j(A, B)$\\ \hline
		t & t & \pause  t & \pause $A \land B$\\ \hline
		t & f & \pause f &\pause --- \\ \hline
		f & t & \pause t & \pause $\lnot A \land B$ \\ \hline
		f & f & \pause t & \pause $\lnot A \land \lnot B$
	\end{tabular}
	\vspace{1.5em}
	
	Therefore, $P$ could be rewritten in disjunctive normal form as $P^{\prime}(A, B) = (A \land B) \lor (\lnot A \land B) \lor (\lnot A \land \lnot B)$.
	\end{center}
\end{example}
\end{frame}

\section{Minimization of Propositional Logic Formulas}
\subsection{KV-Diagrams -- Motivation}
\begin{frame}

	\frametitle{Minimizing Formulas -- Motivation}
	
	\begin{itemize}
		\item Easier to understand
		\item Save money when constructing integrated circuits
	\end{itemize}
\end{frame}

\begin{frame}
	\frametitle{KV-Diagrams}
	Consider the following formula in DNF
	\[ P^{\prime}(A, B) = (A \land B) \lor (\lnot A \land B) \lor (\lnot A \land \lnot B).\]
	
	In the two dimensional case we construct a diagram in the following way:
	
	\vspace{1.5em}
	
	\begin{center}
	\begin{tabular}{c|c|c|c}
		& $\lnot A$ & $A$ \\ \hline
		$\lnot B$ & &  \\ \hline
		$B$ & &   \\ \hline
	\end{tabular}
	\end{center}
\end{frame}

\begin{frame}
	\frametitle{KV-Diagrams}
	Consider the following formula in DNF
	\[ P^{\prime}(A, B) = \colorbox{red}{($A \land B$)} \lor (\lnot A \land B) \lor (\lnot A \land \lnot B).\]
	
	In the two dimensional case we construct a diagram in the following way:
	
	\vspace{1.5em}
	
	\begin{center}
	\begin{tabular}{c|c|c|c}
		& $\lnot A$ & $A$ \\ \hline
		$\lnot B$ & &  \\ \hline
		$B$ & &  \\ \hline
	\end{tabular}
	\end{center}
\end{frame}

\begin{frame}
	\frametitle{KV-Diagrams}
	Consider the following formula in DNF
	\[ P^{\prime}(A, B) = \colorbox{red}{($A \land B$)} \lor (\lnot A \land B) \lor (\lnot A \land \lnot B).\]
	
	In the two dimensional case we construct a diagram in the following way:
	
	\vspace{1.5em}
	
	\begin{center}
	\begin{tabular}{c|c|c|c}
		& $\lnot A$ & $A$ \\ \hline
		$\lnot B$ & &  \\ \hline
		$B$ & &  \colorbox{red}{1} \\ \hline
	\end{tabular}
	\end{center}
\end{frame}

\begin{frame}
	\frametitle{KV-Diagrams}
	Consider the following formula in DNF
	\[ P^{\prime}(A, B) = (A \land B) \lor \colorbox{red}{$(\lnot A \land B)$} \lor (\lnot A \land \lnot B).\]
	
	In the two dimensional case we construct a diagram in the following way:
	
	\vspace{1.5em}
	
	\begin{center}
	\begin{tabular}{c|c|c|c}
		& $\lnot A$ & $A$ \\ \hline
		$\lnot B$ & &  \\ \hline
		$B$ & &  1\\ \hline
	\end{tabular}
	\end{center}
\end{frame}

\begin{frame}
	\frametitle{KV-Diagrams}
	Consider the following formula in DNF
	\[ P^{\prime}(A, B) = (A \land B) \lor \colorbox{red}{$(\lnot A \land B)$} \lor (\lnot A \land \lnot B).\]
	
	In the two dimensional case we construct a diagram in the following way:
	
	\vspace{1.5em}
	
	\begin{center}
	\begin{tabular}{c|c|c|c}
		& $\lnot A$ & $A$ \\ \hline
		$\lnot B$ & &  \\ \hline
		$B$ & \colorbox{red}{1}&  1\\ \hline
	\end{tabular}
	\end{center}
\end{frame}

\begin{frame}
	\frametitle{KV-Diagrams}
	Consider the following formula in DNF
	\[ P^{\prime}(A, B) = (A \land B) \lor (\lnot A \land B) \lor \colorbox{red}{$(\lnot A \land \lnot B)$}.\]
	
	In the two dimensional case we construct a diagram in the following way:
	
	\vspace{1.5em}
	
	\begin{center}
	\begin{tabular}{c|c|c|c}
		& $\lnot A$ & $A$ \\ \hline
		$\lnot B$ & &  \\ \hline
		$B$ & 1 &  1\\ \hline
	\end{tabular}
	\end{center}
\end{frame}

\begin{frame}
	\frametitle{KV-Diagrams}
	Consider the following formula in DNF
	\[ P^{\prime}(A, B) = (A \land B) \lor (\lnot A \land B) \lor \colorbox{red}{$(\lnot A \land \lnot B)$}.\]
	
	In the two dimensional case we construct a diagram in the following way:
	
	\vspace{1.5em}
	
	\begin{center}
	\begin{tabular}{c|c|c|c}
		& $\lnot A$ & $A$ \\ \hline
		$\lnot B$ & \colorbox{red}{1} &  \\ \hline
		$B$ & 1 &  1\\ \hline
	\end{tabular}
	\end{center}
\end{frame}

\begin{frame}
	\frametitle{KV-Diagrams}
	Consider the following formula in DNF
	\[ P^{\prime}(A, B) = (A \land B) \lor (\lnot A \land B) \lor (\lnot A \land \lnot B).\]
	
	In the two dimensional case we construct a diagram in the following way:
	
	\vspace{1.5em}
	
	\begin{center}
	\begin{tabular}{c|c|c|c}
		& $\lnot A$ & $A$ \\ \hline
		$\lnot B$ & 1 &  \\ \hline
		$B$ & 1 &  1\\ \hline
	\end{tabular}
	\end{center}
	
	Now we try to find boxes of size 2 or 4 or 8, etc. These boxes should be as large as possible. The number of boxes should be as small as possible. The boxes may overlap.
\end{frame}

\begin{frame}
	\frametitle{KV-Diagrams}
	Consider the following formula in DNF
	\[ P^{\prime}(A, B) = (A \land B) \lor (\lnot A \land B) \lor (\lnot A \land \lnot B).\]
	
	In the two dimensional case we construct a diagram in the following way:
	
	\vspace{1.5em}
	
	\begin{center}
	\begin{tabular}{c|c|c|c}
		& $\lnot A$ & $A$ \\ \hline
		$\lnot B$ & \colorbox{red}{1} &  \\ \hline
		$B$ & \colorbox{red}{1} &  1\\ \hline
	\end{tabular}
	\end{center}
	
	$\lnot A$
\end{frame}

\begin{frame}
	\frametitle{KV-Diagrams}
	Consider the following formula in DNF
	\[ P^{\prime}(A, B) = (A \land B) \lor (\lnot A \land B) \lor (\lnot A \land \lnot B).\]
	
	In the two dimensional case we construct a diagram in the following way:
	
	\vspace{1.5em}
	
	\begin{center}
	\begin{tabular}{c|c|c|c}
		& $\lnot A$ & $A$ \\ \hline
		$\lnot B$ & 1 &  \\ \hline
		$B$ & \colorbox{red}{1} &  \colorbox{red}{1}\\ \hline
	\end{tabular}
	\end{center}
	
	$\lnot A \lor B$
\end{frame}

\begin{frame}
	\frametitle{KV-Diagrams -- Another Example}
	\begin{example}
		\begin{align*}
			(\lnot A \land B \land C \land \lnot D) & \lor (A \land \lnot B \land \lnot C \land \lnot D) & \lor (A \land \lnot B \land \lnot C \land D) & \lor \\
			(A \land \lnot B \land C \land D) & \lor (A \land \lnot B \land C \land \lnot D) & \lor (A \land B \land \lnot C \land \lnot D) & \lor \\
			(A \land B \land \lnot C \land D) & \lor (A \land B \land C \land \lnot D)
		\end{align*}
	\end{example}
\end{frame}

\begin{frame}
	\frametitle{Another Example -- Sample Solution}
	\pause
	
	\begin{center}
	\begin{tabular}{c|c|c|c|c|c}
	 & $A$ & $A$ & $\lnot A$ & $\lnot A$ & \\ \hline
	$B$ & & 1 & & & $D$ \\ \hline
	$B$ & 1 & 1 & & 1 & $\lnot D$ \\ \hline
	$\lnot B$ & 1 & 1 & & & $\lnot D$ \\ \hline
	$\lnot B$ & 1 & 1 & & & $D$ \\ \hline
	 & $C$ & $\lnot C$ & $\lnot C$ & $C$ &
	\end{tabular}
	\end{center}
	
	\vspace{1.5em}
	
	$(A \land \lnot C) \lor (A \land \lnot B) \lor (B \land C \land \lnot D)$
\end{frame}

\begin{frame}
	\frametitle{KV-Table for Three Variables}
	
	\begin{center}
	\begin{tabular}{c|c|c|c|c|c}
	 & $A$ & $A$ & $\lnot A$ & $\lnot A$ & \\ \hline
	 $B$ & &&&& \\ \hline
	 $\lnot B$ &&&&& \\ \hline
	  & $C$ & $\lnot C$ & $\lnot C$ & $C$ &
	
	\end{tabular}
	\end{center}
\end{frame}

\subsection{Quine-McCluskey Algorithm}
\begin{frame}
\frametitle{Minterm and Literal}

\begin{definition}
	Given a propositional logic formula $P = Q_1 \lor Q_2 \lor \ldots \lor Q_m$, where all $Q_j (1 \leq j \leq m)$ are of the form
	\[Q_j = \left ( \left ( \begin{matrix}A_1 \\ \text{or} \\ \lnot A_1  \end{matrix} \right ) \land
		\left ( \begin{matrix}A_2 \\ \text{or} \\ \lnot A_2  \end{matrix} \right ) \land \ldots \land
		\left (\begin{matrix}A_n \\ \text{or} \\ \lnot A_n  \end{matrix}\right ) \right ) \]

	
	we call $Q_j$ a {\em minterm}. The $A_i$ or $\lnot A_i$ ($1 \leq i \leq n$) are called {\em literals}.
\end{definition}

\pause

\begin{remark}
Sometimes we also call a row of a truth table which result is true a minterm.
\end{remark}

\pause

\begin{remark}
In case of a multiple appearance of a literal $A_i$ in a $Q_j$, $Q_j$ is called a product term.
\end{remark}
\end{frame}

\begin{frame}
\frametitle{Implicant}

\begin{definition}
Given a propositional logic formula $P = Q_1 \lor \ldots \lor Q_m$ with each minterm $Q_i$ having $n$ literals $A_1, A_2, \ldots, A_n$. Furthermore we have a product term
\[I = \left ( \left ( \begin{matrix}A_1 \\ \text{or} \\ \lnot A_1  \end{matrix} \right ) \land
		\left ( \begin{matrix}A_2 \\ \text{or} \\ \lnot A_2  \end{matrix} \right ) \land \ldots \land
		\left (\begin{matrix}A_n \\ \text{or} \\ \lnot A_n  \end{matrix}\right ) \right ).\]

\pause

$I$ is called a {\em prime implicant} of $P$ if $I \Rightarrow P$, i.e., that whenever the evaluation of $I$ with a specific set of truth values for $A_1, A_2, \ldots, A_n$ yields true, the evaluation of $P$ with the same set of truth values also yields true.
\end{definition}
\end{frame}


\begin{frame}
	\frametitle{Example and Remark}
	
	Consider the following propositional logic formula $P$:
	
	\vspace{1em}
	
	\begin{tabular}{ccc}
	$P$ & = & $(A_1 \land A_2) \lor (A_2 \land A_3) \lor A_4$
	\end{tabular}
	
	\vspace{1em}
	
	This formula is implied by the formulas $A_1 \land A_2$, $A_2 \land A_3$, $A_4$, $A_1 \land A_2 \land A_3$, $A_1 \land A2_ \land A_4$ and many others. These are the implicants of $P$.
	
	\pause
	\begin{remark}
		Implicants are sometimes called ``coverings" of one or more minterms.
	\end{remark}
\end{frame}

\begin{frame}
\frametitle{Prime Implicant}

\begin{definition}
Given a propositional logic formula $P$. An implicant $I$ is called a {\em prime implicant} if it can't be reduced any further, i.e., any removal of a literal would make it a ``non-implicant".
\end{definition}

\pause

\begin{example}
	Let 
	\begin{tabular}{ccc}
	$P$ & = & $(A_1 \land A_2) \lor (A_2 \land A_3) \lor A_4$
	\end{tabular}
	be a propositional logic formula. The prime implicants of $P$ would be $A_1 \land A_2$, $A_2 \land A_3$, and $A_4$ would be the prime implicants of $P$.

\end{example}
\end{frame}

\begin{frame}
\frametitle{Prime Implicant and Minimal Formulas}
\begin{definition}
	A disjunction of prime implicants covering a complete formula $P$ is called a minimal formula of $P$.
\end{definition}

\begin{remark}
	There may exist more than one minimal formula.
\end{remark}
\end{frame}

\begin{frame}
\frametitle{Example}
Given the following truth table $M$:
\begin{center}
	\begin{tabular}{c|c|c||c}
		$x_1$ & $x_2$ & $x_3$ & $y$ \\ \hline
		f & f & f & f \\ \hline
		f & f & t & f \\ \hline
		f & t & f & f \\ \hline
		f & t & t & t \\ \hline
		t & f & f & t \\ \hline
		t & f & t & t \\ \hline
		t & t & f & f \\ \hline
		t & t & t & t
	\end{tabular}
\end{center}
\end{frame}

\begin{frame}
\frametitle{Example continued}
\begin{itemize}

\item
\begin{tabular}{ccccc}
$\lnot A_1$ & $\land$ & $A_2$ & $\land$ & $A_3$ \\
$A_1$ &$\land$& $\lnot A_2$ &$\land$& $\lnot A_3$ \\
$A_1$ & $\land$ & $\lnot A_2$ & $\land$ & $A_3$  \\
$A_1$ & $\land$ & $A_2$ & $\land$ & $A_3$
\end{tabular}

are the  minterms of $M$.

\pause
\item $A_1 \land \lnot A_2$ and $A_2 \land A_3$ are prime implicants of $M$.

\pause
\item $A_1 \land \lnot A_2 \lor A_2 \land A_3$ is a minimal formula representing $M$.
\end{itemize}
\end{frame}

\begin{frame}
\frametitle{Quine-McCluskey Algorithm}
\begin{itemize}
	\item An algorithm to find minimal formulas
	\item Finds all prime implicants of a given formula $S$
	\item Finds a combination of prime implicants which covers $S$
	\item Example as given in class
\end{itemize}
\end{frame}

\end{document}