\documentclass{beamer}
%%% BEGIN PREAMBLE

\usepackage{amsmath}
\usepackage{amsfonts}
% \usepackage{beamerthemesplit} // Activate for custom appearance
\usepackage{graphicx}
\usepackage[utf8]{inputenc}

% Definitions
\theoremstyle{remark}
\newtheorem{remark}[theorem]{Remark}

% Title stuff
\title{Predicate Logic}
\author{Peter Bauer}
\date{} % delete this line to display the current date

%%% BEGIN DOCUMENT
\begin{document}

\frame{\titlepage}

\begin{frame}
	\frametitle{Outline}
	\tableofcontents
\end{frame}

\section{Terminology}
\begin{frame}
\frametitle{Motivation}
\begin{itemize}
	\item Propositional logic deals with propositions.
	\pause
	\item Propositions were considered to be "atomic"
	\pause
	\item To express propositions more concretely we need more sophisticated terms
	\pause
	\item Examples
	\begin{itemize}
		\item ``$x$ is a prime number.''
		\item ``4 is an even number.''
		\item ``This shark is 20 ft. long.''
	\end{itemize}
\end{itemize}	
	
\end{frame}

\begin{frame}
\frametitle{Variable}
\begin{definition}
A {\em variable} is a name for an arbitrary element out of a universe.
\end{definition}

\pause
\begin{remark}
\begin{itemize}
 	\item Variables denote places where values out of the universe can be seen
	\pause
	\item Variables are assigned specific values out of the universe
	\pause
	\item Names of variables are combinations of characters, numbers and other symbols.
\end{itemize}

\end{remark}
\end{frame}

\begin{frame}
 \frametitle{Constant}
 If  we want to consider concrete elements of a universe we use constants.
 \pause
\begin{definition}
{\em Constants} are names of concrete objects. These can be concrete values (e.g., $\pi$, 17, 42), function constants (e.g., $+, \cos, \sqrt{\hspace{.5em} }$) or predicate constants (e.g., $\geq, \in$).
\end{definition}
\end{frame}

\begin{frame}
\frametitle{Term}
\begin{definition}
Constants and variables are {\em terms}. Syntactic structures of the form $f(t_1, t_2, \ldots, t_n)$ are {\em terms}, if $f$ is a function constant with arity $n$ and $t_1, t_2, \ldots, t_n$ are terms.
\end{definition}

\pause

\begin{example}
\begin{itemize}
	\item $\sin(x)$, where $\sin$ is the function constant with arity 1
	\pause
	\item $x + y$, where $+$ is the function constant with arity 2
	\pause
	\item $\sqrt{y}$, where $\sqrt{\hspace{.5em}}$ is the function constant with arity 1
\end{itemize}
\end{example}

\pause
\begin{remark}
 Constant values like $\pi$, 17, or ``Vergissmeinnicht'' can be interpreted as terms. These are terms with a functions constant of arity 0.
\end{remark}
\end{frame}

\begin{frame}
\frametitle{Predicate and Formula}

\begin{definition}
Phrases which describe properties of objects or relations between objects are called {\em predicate constants} or sometimes {\em predicates}.\end{definition}

\pause
\begin{example}
	``is prime'', ``{\tt isValidInput}'', $\geq, =$ are predicate constants.
\end{example}
\pause

\begin{definition}
Syntactic structures of the form $P(t_1, t_2, \ldots, t_n)$ are {\em atomic formulas}, if $P$ is a predicate constant with arity $n$ and $t_1, t_2, \ldots t_n$ are terms.
\end{definition}

\end{frame}

\begin{frame}
\frametitle{Some Examples and Conventions}
\begin{remark}
Instead of the bloomy phrases of everyday life language, we use very slim and symbolic structures in mathematical logic
\end{remark}
\pause
\begin{example}
\begin{tabular}{|c|c|}
\hline
everyday language & logic \\ \hline
``$y$ is at least as great as $z$'' & $y \geq z$ \\
``$a$ leads project $b$'' & ProjectLead($a, b$) \\ \hline
 \end{tabular}
\end{example}
 \pause
\begin{remark}
\begin{itemize}
	\item The symbols $a, b, y$, and $z$ are variables, ``$\geq$'' and ``ProjectLead'' are predicate constants.
	\pause
	\item In the sequel, we will use upper case letters like $P, Q, R, \ldots$ to denote predicate constants.
	\pause
	\item $P, Q, R, \ldots$ are always predicate constants and are {\em not} predicate variables (first order predicate logic)
\end{itemize}
\end{remark}
\end{frame}

\begin{frame}
\frametitle{Another Example}
Combining predicate formulas by means of compositions as given in propositional logic lead again to formulas of predicate logic.
\pause
\begin{example}
\begin{itemize}
	\item Let $x$ be a student of the HTL Leonding.
	\pause
	\item Let $P(x) = $ ``$x$ studies LOAL'' and $Q(x) = $ ``$x$ is a first grader''.
	\pause
	\item Then the conjunction $P(x) \land Q(x)$ is again a formula of predicate logic.
	\pause
	\item $P(x) \land Q(x) = $ ``$x$ studies LOAL and is a first grader''.
\end{itemize}

\end{example}
\end{frame}

\section{Quantifiers}
\begin{frame}
\frametitle{Quantifier --- Motivation}
\begin{itemize}
	\item The truth value of atomic formulas can only be determined if we assign constants to each variable.
	\item It can't be determined in general, in particular, if the formula contains free variables.
\end{itemize}

\pause

\begin{example}
\begin{itemize}
 	\item Consider the binary predicate constant $\geq$ over $\mathbb{N}$.
	\pause
	\item Consider further the formula $a \geq 3$.
	\pause
	\item $a$ is a free variable and 3 is a constant.
	\pause
	\item Now consider the formula $7 \geq 3$.
	\pause
	\item Furthermore, consider the formula ``for all $a \in \mathbb{N}$ holds: $a \geq 3$''
\end{itemize}
\end{example}
\end{frame}

\begin{frame}
\frametitle{Quantifier}
\begin{itemize}
	\item The last example has shown that we need phrases like ``for all variables \ldots'' or ``there exists \ldots''.
	\pause
	\item For this purpose we have {\em Quantifiers} in mathematical logic.
	\pause
	\item Quantifiers bind variables in formulas of predicate logic.
\end{itemize}
\end{frame}

\begin{frame}
\frametitle{Quantifier --- Definition}
\begin{definition}
The most important quantifiers are the {\em universal quantifier} and the {\em existential quantifier} which are written and spoken as follows:
\begin{enumerate}
	\item Universal Quantifier $\forall$
	\begin{itemize}
		\item $\forall x: P(x)$
		\pause
		\item for all $x$ holds $P$
	\end{itemize}
	\pause
	\item Existential Quantifer $\exists$
		\begin{itemize}
			\item $\exists x: P(x)$
			\pause
			\item There exists (at least one) $x$ for which holds $P$
		\end{itemize}
\end{enumerate}
\end{definition}

\pause

\begin{remark}
In the sequel we will use the notations $\forall x: P(x)$ and $\exists x: P(x)$.
\end{remark}
\end{frame}

\begin{frame}
\frametitle{Free and Bound Variables}
\begin{itemize}
	\item Variables which are referred to by a quantifier are called {\em bound variables}.
	\item All other variables are {\em free variables} unless they are bound by another quantifier.
\end{itemize}

\pause
\begin{example}
\begin{itemize}
	\item Consider the formula ``$x$ has birthday in the same month as $y$''.
	\item Let $x$ and $y$ denote students of the HTL Leonding.
	\item $x$ and $y$ are both free variables.
	\pause
	\item $\exists x: x$ has birthday in the same month as $y$.
	\item $x$ is a bound variable and $y$ is a free variable.
\end{itemize}
\end{example}
\end{frame}

\begin{frame}
\frametitle{Quantified Formulas}
\begin{enumerate}
	\item The formula $\forall x: P(x)$ is true if and only if $P$ is true for every value of the universe that is assigned to $x$.
	\pause
	\item The formula $\exists x: P(x)$ is true if and only if $P$ is true for at least one value of the universe that is assigned to $x$.
	\pause
	\item The formulas $\forall x: P(x, y, z)$ and $\exists x: P(x, y, z)$ are formulas with free variables $y$ and $z$. Therefore, no
	truth value can be determined for these formulas.
\end{enumerate}
\pause
\begin{remark}
\begin{itemize}
	\item Since we can determine a truth value for the first and the second formula, we can call these formulas propositions.
	\pause
	\item The third remains a formula of predicate logic.
\end{itemize}
\end{remark}
\end{frame}

\begin{frame}
 \frametitle{Example}
 \begin{example}
 
\begin{itemize}
	\item Let's consider the universe of all students of the HTL Leonding.
        \item Further consider the formula ``$x$ attends the 3\textsuperscript{rd} grade''.
	\pause
	\item This formula has one free variable $x$.
	\pause
	\item ``$\forall x: $$x$ attends the 3\textsuperscript{rd} grade'' would only be true if all students of this school would attend the 3\textsuperscript{rd} grade.
	\item ``$\exists x: $$x$ attends the 3\textsuperscript{rd} grade'' is true, since there exists at least one student at this school who attends the 3\textsuperscript{rd} grade.
\end{itemize}
\end{example}
\end{frame}

\begin{frame}
 \frametitle{Summary}
 
 We know two possibilities to turn a formula of predicate logic into a proposition 
\begin{enumerate}
 	\item Assignment of constants out of the universe to each free variable.
	\item Binding the free variables with quantifiers.
	\pause
	\item Formulas with no free variables are propositions and are sometimes called {\em closed formulas}.
	\item Only these have a truth value.
\end{enumerate}
\end{frame}

\begin{frame}
	\frametitle{Relationships Between Negation and Quantifiers}
	
	\begin{theorem}
		Let $P$ be a unary predicate constant and $x \in D$ a variable. Then the following relations hold:
		\begin{align*}
			\lnot \exists x: P(x) & \equiv \forall x: \lnot P(x) \\
			\lnot \forall x: P(x) & \equiv \exists x: \lnot P(x)
		\end{align*}
	\end{theorem}
	
	\pause
	
	\begin{example}
		\begin{align*}
			\lnot \forall x \in \mathbb{N}: \text{isPrime}(x) \Rightarrow \text{odd}(x) \\
			\exists x \in \mathbb{N}: \text{isPrime}(x) \land \lnot \text{odd}(x)
		\end{align*}
	\end{example}
\end{frame}
\end{document}
